\documentclass[12pt]{article}

%Paquetes
\usepackage[left=2cm,right=2cm,top=3cm,bottom=3cm,letterpaper]{geometry}
\usepackage{lmodern}
\usepackage[T1]{fontenc}
\usepackage[utf8]{inputenc}
\usepackage[spanish,activeacute]{babel}
\usepackage{mathtools}
\usepackage{amssymb}
\usepackage{enumerate}
\usepackage{tabularx}
\usepackage{wasysym}
\usepackage{listings}
\usepackage{graphicx}
\usepackage{hyperref}
%\usepackage{graphicx}
%\graphicspath { {tarea01/media/} }
%\usepackage{pifont}

%Preambulo
\title{Tecnologías para desarrollos en internet \\ Manual CRUD: Beego}
\author{Kihui-DEV}
\date{Fecha: 18/09/16 \\ Facultad de Ciencias UNAM}

\begin{document}
\maketitle
\tableofcontents{}
\newpage

\section{Introducción}
\section{Instalación de Go 1.7}
Paquetes de instalación para \textit{Apple OS X, Microsoft Windows y Linux} son provistos en la página oficial de descargas de \href{https://golang.org/dl/}{Go}. También viene incluido entre las opciones el código fuente del compilador del lenguaje junto con instrucciones para llegar a una instalación tan completa como las demás. \par
A continuación, presentamos la instalación para \textit{Linux y OS X}. \footnote{Si se desea revisar la configuración para \textit{Windows}, hacer uso del siguiente enlace:
  \href{https://golang.org/doc/install?download=go1.7.1.windows-amd64.msi}{MSI Installer}}

\subsection{Descarga}\label{sec:d}


\subsubsection*{Linux}
Para obtener el paquete de Golang\footnote{Actualmente la versión 1.7.1} \\
Ejecutar:
\begin{verbatim}
  $ wget https://golang.org/doc/install?download=go1.7.1.linux-amd64.tar.gz
\end{verbatim}
O bien descargarlo directamente desde este enlace:
\begin{center}
\href{https://golang.org/doc/install?download=go1.7.1.linux-amd64.tar.gz}{Go 1.7 para Linux y OS X}.
\end{center}


\subsubsection*{Mac OS X}
Alternativa a las siguientes instrucciones, existe la opción de descargar el fichero \textit{.pkg} instalable de \textit{Go} que automatiza la configuración para este sistema operativo.\footnote{\href{https://golang.org/doc/install?download=go1.7.1.darwin-amd64.pkg}{Instalador para Mac OS X}}\\[1mm]

Para obtener una configuración inicial personalizada \textit{Mac OS X} utilice el paquete descargable para \textit{Linux} disponible en la sección anterior.\\

El resto de la configuración se sigue de la misma manera para ambas plataformas.

\subsection{Configuración}\label{sec:c}
Primero extraemos los archivos del paquete comprimido sobre algún directorio.\\
Para descomprimir el paquete sobre \textit{/usr/local} como es usual:
\begin{verbatim}
  # tar -C /usr/local -xzf go1.7.1.linux-amd64.tar.gz
\end{verbatim}
Al finalizar la extracción, procedemos a establecer una ruta a los binarios de las herramientas de \textit{Go}, y así tenerlas disponibles en cada sesión.\par
Agregar esta línea en el script de inicio (típicamente sobre \textit{/user/local/profile}
si se desea hacer una instalación general en el sistema operativo o
en particular para el usuario en curso, usar en cambio \textit{\char126/.profile}):
\begin{verbatim}
  export PATH=$PATH:/usr/local/go/bin
\end{verbatim}

\subsection{Prueba}
Crear un directorio que haga de workspace para la prueba.\\
Por ejemplo:
\begin{verbatim}
  $ mkdir ~/go
\end{verbatim}
Asignar la variable \textbf{GOPATH}\footnote{La misma línea puede agregarse al script de inicio \textit{profile} manejado en la sección de configuracion para evitar ejecutarla y mantener los proyectos y aplicaciones de \textit{Go} ubicados en un sólo directorio.}
para que apunte a tal dirección:
\begin{verbatim}
  $ export GOPATH=$HOME/go
\end{verbatim}
Bien podemos hacer persistente este cambio agregando la misma línea al script
de inicio que editamos en la sección anterior (ir a Descarga~\ref{sec:d} y configuración~\ref{sec:c}). \par

A continuación, creamos dentro de ese directorio  \textit{src/hola}.
Y dentro de \textit{hola/} un fichero nuevo de nombre \textit{hola.go}:
\begin{verbatim}
     package main

     import "fmt"

     func main() {
         fmt.Printf("hola, mundo\n")
     }
\end{verbatim}

Luego, desde cualquier ubicación podemos ejecutar\\
esto:
\begin{verbatim}
  $ go install hola
\end{verbatim}

Esto producirá un ejecutable \textit{hola} dentro de el directorio \textit{go/bin/},
que podemos ejecutar utilizando lo siguiente:
\begin{verbatim}
  $ $GOPATH/bin/hola
\end{verbatim}

o directamente sobre el directorio donde se encuentre el ejecutable:

\begin{verbatim}
  $ ./hola
\end{verbatim}

Si produce la salida ``hola, mundo'', quiere decir que nuestra instalación fue exitosa. \newpage


\section{Instalación de Beego}

\noindent Para instalar la última versión de Beego\footnote{A la fecha de elaboración de este manual: 1.7.1} utilizamos el siguiente comando:
\begin{verbatim}
  $ go get github.com/astaxie/beego
\end{verbatim}

\noindent Para compilar y correr nuestros proyectos necesitaremos instalar Bee\footnote{A la fecha de elaboración de este manual: 1.5.2} también :
\begin{verbatim}
  $ go get github.com/beego/bee
  $ go install github.com/beego/bee
\end{verbatim}

Para poder utilizar \textit{bee} sin necesidad de ir a la carpeta de binarios de \textit{Go}, podemos crear un enlace simbólico que apunte precisamente al ejecutable.
\begin{verbatim}
  # ln -s $GOPATH/bin/bee /usr/bin/bee
\end{verbatim}

\section{Creación de proyecto}\label{sec:proy}
\noindent Para crear un proyecto en Beego, necesitamos ir al directorio de nuestro $\$$GOPATH, donde escribimos el siguiente comando:
\begin{verbatim}
$ bee new beego-crud
\end{verbatim}
\noindent Podremos ver que se han creado las siguientes carpetas y archivos necesarios para nuestra aplicación:

\begin{verbatim}
beego-crud/
|-- conf/
|   |__ app.conf
|-- controllers/
|   |__ default.go
|-- main.go
|-- models/
|-- routers/
|   |__ router.go
|-- static/
|   |--- css/
|   |--- img/
|   |___ js/
|-- tests/
|   |__ default_test.go
|-- views/
    |__ index.tpl
\end{verbatim}

\subsection{Estructura del proyecto - MVC}
\begin{enumerate}[1)]
  \subsubsection*{Modelo}
\item \textit{conf/}
\item \textit{models/}
  \subsubsection*{Vista}
\item \textit{controllers/}
\item \textit{routers/}
  \subsubsection*{Controlador}
\item \textit{views/}
  \subsubsection*{Extra}
\item \textit{static/}
\item \textit{tests/}

\end{enumerate}


\subsection{Prueba del servidor local}
\noindent Finalmente para correr el nuevo proyecto que hemos creado, hacemos lo siguiente:
\begin{verbatim}
$ cd $GOPATH/src/beego-crud
$ bee run
\end{verbatim}
\noindent Ingresamos \textit{localhost:8080} como dirección en el navegador para encontrarnos con la siguiente página de inicio: \\

\includegraphics[scale=0.25]{beego.png}
\newpage

\section{Elaboración del CRUD bajo MVC}
En esta sección revisaremos la implementación de un CRUD (\textit{Create-Read-Update-Delete}) sobre la aplicación que creamos en la sección anterior (s.~\ref{sec:proy}). Comenzaremos explicando los documentos que tenemos que modificar para tener una configuración exitosa y la forma de escribir el código de acuerdo al MVC (\textit{Modelo Vista Controlador}), siguiendo la estructura que nos proporciona \textit{Beego} para éste.

\subsection{Preparación}
Para implementar nuestro primer CRUD en Beego es necesario tener un sistema de
datos persistente sobre el
que realicemos nuestras operaciones.
La mejor solución que en casi todos los casos se señala para lograr esto en el desarrollo
web por supuesto no son los archivos, sino una base de datos, por lo general
relacional. \par
Todo proyecto de Beego, como ya vimos, está organizado segun el MVC. Es precisamente
sobre el archivo de \textbf{models.go} que escribiremos nuestras
relaciones expresadas con atributos asignados de acuerdo a los tipos de dato
que maneja Go para después verlos reflejados en tablas en una base datos ya explorable.
Bien podría ser en sentido inverso, desde un esquema \textit{SQL} generar los modelos
para nuestro proyecto de Beego, pero en nuestro caso decidimos implementarlo
en esta dirección porque nuestra intención es enfocarnos en Beego, no en SQL.\par
A continuación presentamos la configuración de la base de datos y un mapeo
de una sola relación con la que trabajaremos todas las operaciones.
\subsubsection*{Base de datos}
\textit{Beego-ORM} es la herramienta de Mapeo Objeto-Relacional de \textit{Beego}
escrita en \textit{Go}. Según la página oficial está inspirada en \textit{Django ORM}
y \textit{SQLAlchemy}. Se advierte que al estar en desarrollo, no se garantiza
un 100\% de compatibilidad y es posible encontrar algunos ``bugs'', que el
equipo de desarrollo de Beego promete solucionar apenas se reporten.\\

Lo primero que hay que considerar es la conexión con la base de datos que necesitamos para nuestro modelo.
Beego provee soporte para tres sistemas manejadores de bases de datos
con sus respectivos ``drivers'':
\begin{itemize}
\item \href{http://www.mysql.com/}{MySQL}
\item \href{https://sqlite.org/}{Sqlite}
\item \href{http://www.postgresql.org.es/}{Postgres}
\end{itemize}

Por su sencilla configuración y mayoría de ejemplos de implementación junto con
Beego, escogimos MySQL por lo que los siguientes puntos se ejemplificarán
con el supuesto de que trabajamos con este SMBD\footnote{Para revisar más ampliamente
  las opciones de bases de datos recomendamos visitar este apartado de la documentación:
  \href{http://beego.me/docs/mvc/model/orm.md#set-up-database}{ORM Usage}} y que ya se cuenta con una instalación
funcional\footnote{En el manual de MySQL v5.7 viene una amplia guía de instalación.
  Recomendamos los siguientes apartados para distintos sistemas operativos:
  \href{http://dev.mysql.com/doc/refman/5.7/en/windows-installation.html}{Windows} |
  \href{http://dev.mysql.com/doc/refman/5.7/en/osx-installation.html}{OS X} |
  \href{http://dev.mysql.com/doc/refman/5.7/en/linux-installation.html}{Linux}}.\\[2mm]
Lo primero que haremos será crear nuestra base de datos:
\begin{verbatim}
  $ mysql -u root -p
  MariaDB [(none)]> create database beego;
  MariaDB [(none)]> exit
\end{verbatim}
Si no experimentamos ninguna dificultad con esto, podemos continuar con la
configuración de nuestro proyecto.
\subsubsection*{Configuración}
Para realizar la configuración de conf/app.conf~\ref{}

\subsection{Creación}

\subsubsection*{}

\subsection{Lectura}

\subsection{Actualización}

\subsection{Eliminación}

\subsection{Resultados}

\section{Referencias}
\end{document}
