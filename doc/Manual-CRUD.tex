\documentclass[12pt]{article}

%Paquetes
\usepackage[left=2cm,right=2cm,top=3cm,bottom=3cm,letterpaper]{geometry}
\usepackage{lmodern}
\usepackage[T1]{fontenc}
\usepackage[utf8]{inputenc}
\usepackage[spanish,activeacute]{babel}
\usepackage{mathtools}
\usepackage{amssymb}
\usepackage{enumerate}
\usepackage{tabularx}
\usepackage{wasysym}
\usepackage{listings}
\usepackage{graphicx}
\usepackage{hyperref}

%\usepackage{graphicx}
%\graphicspath { {tarea01/media/} }
%\usepackage{pifont}

%Preambulo
\title{Tecnologías para desarrollos en internet \\ Manual CRUD: Beego}
\author{Kihui-DEV}
\date{Fecha: 18/09/16 \\ Facultad de Ciencias UNAM}

\begin{document}
\maketitle
\tableofcontents{}
\newpage

\section{Introducción}
\section{Instalación de Go 1.7}
Paquetes de instalación para \textit{Apple OS X, Microsoft Windows y Linux} son provistos en la página oficial de descargas de \href{https://golang.org/dl/}{Go}. También viene incluido entre las opciones el código fuente del compilador del lenguaje junto con instrucciones para llegar a una instalación tan completa como las demás. \par
A continuación, presentamos la instalación para \textit{Linux y OS X}. \footnote{Si se desea revisar la configuración para \textit{Windows}, hacer uso del siguiente enlace:
  \href{https://golang.org/doc/install?download=go1.7.1.windows-amd64.msi}{MSI Installer}}

\subsection{Descarga}\label{sec:d}


\subsubsection*{Linux}
Para obtener el paquete de Golang 1.7 \\
Ejecutar:
\begin{verbatim}
  $ wget https://golang.org/doc/install?download=go1.7.1.linux-amd64.tar.gz
\end{verbatim}
O bien descargarlo directamente desde este enlace:
\begin{center}
\href{https://golang.org/doc/install?download=go1.7.1.linux-amd64.tar.gz}{Go 1.7 para Linux y OS X}.
\end{center}


\subsubsection*{Mac OS X}
Alternativa a las siguientes instrucciones, existe la opción de descargar el fichero \textit{.pkg} instalable de \textit{Go} que automatiza la configuración para este sistema operativo.\footnote{\href{https://golang.org/doc/install?download=go1.7.1.darwin-amd64.pkg}{Instalador para Mac OS X}}\\[1mm]

Para obtener una configuración inicial personalizada \textit{Mac OS X} utilice el paquete descargable para \textit{Linux} disponible en la sección anterior.\\

El resto de la configuración se sigue de la misma manera para ambas plataformas.

\subsection{Configuración}\label{sec:c}
Primero extraemos los archivos del paquete comprimido sobre algún directorio.\\
Para descomprimir el paquete sobre \textit{/usr/local} como es usual:
\begin{verbatim}
  # tar -C /usr/local -xzf go1.7.1.linux-amd64.tar.gz
\end{verbatim}
Al finalizar la extracción, procedemos a establecer una ruta a los binarios de las herramientas de \textit{Go}, y así tenerlas disponibles en cada sesión.\par
Agregar esta línea en el script de inicio (típicamente sobre \textit{/user/local/profile}
si se desea hacer una instalación general en el sistema operativo o
en particular para el usuario en curso, usar en cambio \textit{\char126/.profile}):
\begin{verbatim}
  export PATH=$PATH:/usr/local/go/bin
\end{verbatim}

\subsection{Prueba}
Crear un directorio que haga de workspace para la prueba.\\
Por ejemplo:
\begin{verbatim}
  $ mkdir ~/go
\end{verbatim}
Asignar la variable \textbf{GOPATH}\footnote{La misma línea puede agregarse al script de inicio \textit{profile} manejado en la sección de configuracion para evitar ejecutarla y mantener los proyectos y aplicaciones de \textit{Go} ubicados en un sólo directorio.}
para que apunte a tal dirección:
\begin{verbatim}
  $ export GOPATH=$HOME/go
\end{verbatim}
Bien podemos hacer persistente este cambio agregando la misma línea al script
de inicio que editamos en la sección anterior (ir a Descarga~\ref{sec:d} y configuración~\ref{sec:c}). \par

A continuación, creamos dentro de ese directorio  \textit{src/hola}.
Y dentro de \textit{hola/} un fichero nuevo de nombre \textit{hola.go}:
\begin{verbatim}
     package main

     import "fmt"

     func main() {
         fmt.Printf("hola, mundo\n")
     }
\end{verbatim}

Luego, desde cualquier ubicación podemos ejecutar\\
esto:
\begin{verbatim}
  $ go install hola
\end{verbatim}

Esto producirá un ejecutable \textit{hola} dentro de el directorio \textit{go/bin/},
que podemos ejecutar utilizando lo siguiente:
\begin{verbatim}
  $ $GOPATH/bin/hola
\end{verbatim}

o directamente sobre el directorio donde se encuentre el ejecutable:

\begin{verbatim}
  $ ./hola
\end{verbatim}

Si produce la salida ``hola, mundo'', quiere decir que nuestra instalación fue exitosa. \newpage


\section{Instalación de Beego}

\noindent Para instalar Beego utilizamos el siguiente comando:
\begin{verbatim}
  $ go get github.com/astaxie/beego
\end{verbatim}

\noindent Para compilar y correr nuestros proyectos necesitaremos instalar Bee también:
\begin{verbatim}
  $ go get github.com/beego/bee
  $ go install github.com/beego/bee
\end{verbatim}

Para poder utilizar \textit{bee} sin necesidad de ir a la carpeta de binarios de \textit{Go}, podemos crear un enlace simbólico que apunte precisamente al ejecutable.
\begin{verbatim}
  # ln -s $GOPATH/bin/bee /usr/bin/bee 
\end{verbatim}

\section{Creación de proyecto}\label{sec:proy}
\noindent Para crear un proyecto en Beego, necesitamos ir al directorio de nuestro $\$$GOPATH, donde escribimos el siguiente comando:
\begin{verbatim}
$ bee new beego-crud
\end{verbatim}
\noindent Podremos ver que se han creado las siguientes carpetas y archivos necesarios para nuestra aplicación:

\begin{verbatim}
beego-crud/
|-- conf/
|   |__ app.conf
|-- controllers/
|   |__ default.go
|-- main.go
|-- models/
|-- routers/
|   |__ router.go
|-- static/
|   |--- css/
|   |--- img/
|   |___ js/
|-- tests/
|   |__ default_test.go
|-- views/
    |__ index.tpl
\end{verbatim}

\subsection{Estructura del proyecto}
\begin{enumerate}[1)]
\item \textbf{conf/}\label{jiji}
\item \textbf{models/}
\item \textbf{controllers/}
\item \textbf{routers/}
\item \textbf{views/}
\item \textbf{static/}
\item \textbf{tests/}
  
\end{enumerate}


\subsection{Prueba del servidor local}
\noindent Finalmente para correr el nuevo proyecto que hemos creado, hacemos lo siguiente:
\begin{verbatim}
$ cd $GOPATH/src/beego-crud
$ bee run
\end{verbatim}
\noindent Ingresamos \textit{localhost:8080} como dirección en el navegador para encontrarnos con la siguiente página de inicio: \\

\includegraphics[scale=0.25]{beego.png}
\newpage

\section{Elaboración del CRUD bajo MVC}
En esta sección revisaremos la implementación de un CRUD (\textit{Create-Read-Update-Delete}) sobre la aplicación que creamos en la sección anterior (s.~\ref{sec:proy}). Comenzaremos explicando los documentos que tenemos que modificar para tener una configuración exitosa y la forma de escribir el código de acuerdo al MVC (\textit{Modelo Vista Controlador}), siguiendo la estructura que nos proporciona \textit{Beego} para éste.

\subsection{Modelo}
\subsubsection*{Base de datos}
\subsubsection*{Configuración}
Para realizar la configuración de conf/app.conf~\ref{}
\subsubsection*{Modelos}

\subsection{Controlador}
\subsubsection*{Routers}
\subsubsection*{Controladores}

\subsection{Vista}
\subsubsection*{Plantillas}
\subsubsection*{Estilo y agregados}

\subsection{Resultados}

\section{Referencias}
\end{document}
